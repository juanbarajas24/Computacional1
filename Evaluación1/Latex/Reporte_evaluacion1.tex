
\documentclass[11pt]{article}

    
    
    \usepackage[T1]{fontenc}
    
    \usepackage{mathpazo}
	\usepackage[spanish,mexico]{babel}
    \usepackage{cite}
    \usepackage{url}
    \usepackage{graphicx}
    \usepackage{titling}
    \makeatletter
    \def\maxwidth{\ifdim\Gin@nat@width>\linewidth\linewidth
    \else\Gin@nat@width\fi}
    \makeatother
    \let\Oldincludegraphics\includegraphics
   \renewcommand{\includegraphics}[1]{\Oldincludegraphics[width=.8\maxwidth]{#1}}
 
    \usepackage{caption}
    \DeclareCaptionLabelFormat{nolabel}{}
    \captionsetup{labelformat=nolabel}

    \usepackage{adjustbox} % Used to constrain images to a maximum size 
    \usepackage{xcolor} % Allow colors to be defined
    \usepackage{enumerate} % Needed for markdown enumerations to work
    \usepackage{geometry} % Used to adjust the document margins
    \usepackage{amsmath} % Equations
    \usepackage{amssymb} % Equations
    \usepackage{textcomp} % defines textquotesingle
    % Hack from http://tex.stackexchange.com/a/47451/13684:
    \AtBeginDocument{%
        \def\PYZsq{\textquotesingle}% Upright quotes in Pygmentized code
    }
    \usepackage{upquote} % Upright quotes for verbatim code
    \usepackage{eurosym} % defines \euro
    \usepackage[mathletters]{ucs} % Extended unicode (utf-8) support
    \usepackage[utf8x]{inputenc} % Allow utf-8 characters in the tex document
    \usepackage{fancyvrb} % verbatim replacement that allows latex
    \usepackage{grffile} % extends the file name processing of package graphics 
                         % to support a larger range 
    % The hyperref package gives us a pdf with properly built
    % internal navigation ('pdf bookmarks' for the table of contents,
    % internal cross-reference links, web links for URLs, etc.)
    \usepackage{hyperref}
    \usepackage{longtable} % longtable support required by pandoc >1.10
    \usepackage{booktabs}  % table support for pandoc > 1.12.2
    \usepackage[inline]{enumitem} % IRkernel/repr support (it uses the enumerate* environment)
    \usepackage[normalem]{ulem} % ulem is needed to support strikethroughs (\sout)
                                % normalem makes italics be italics, not underlines
    

    
    
    % Colors for the hyperref package
    \definecolor{urlcolor}{rgb}{0,.145,.698}
    \definecolor{linkcolor}{rgb}{.71,0.21,0.01}
    \definecolor{citecolor}{rgb}{.12,.54,.11}

    % ANSI colors
    \definecolor{ansi-black}{HTML}{3E424D}
    \definecolor{ansi-black-intense}{HTML}{282C36}
    \definecolor{ansi-red}{HTML}{E75C58}
    \definecolor{ansi-red-intense}{HTML}{B22B31}
    \definecolor{ansi-green}{HTML}{00A250}
    \definecolor{ansi-green-intense}{HTML}{007427}
    \definecolor{ansi-yellow}{HTML}{DDB62B}
    \definecolor{ansi-yellow-intense}{HTML}{B27D12}
    \definecolor{ansi-blue}{HTML}{208FFB}
    \definecolor{ansi-blue-intense}{HTML}{0065CA}
    \definecolor{ansi-magenta}{HTML}{D160C4}
    \definecolor{ansi-magenta-intense}{HTML}{A03196}
    \definecolor{ansi-cyan}{HTML}{60C6C8}
    \definecolor{ansi-cyan-intense}{HTML}{258F8F}
    \definecolor{ansi-white}{HTML}{C5C1B4}
    \definecolor{ansi-white-intense}{HTML}{A1A6B2}

    % commands and environments needed by pandoc snippets
    % extracted from the output of `pandoc -s`
    \providecommand{\tightlist}{%
      \setlength{\itemsep}{0pt}\setlength{\parskip}{0pt}}
    \DefineVerbatimEnvironment{Highlighting}{Verbatim}{commandchars=\\\{\}}
    % Add ',fontsize=\small' for more characters per line
    \newenvironment{Shaded}{}{}
    \newcommand{\KeywordTok}[1]{\textcolor[rgb]{0.00,0.44,0.13}{\textbf{{#1}}}}
    \newcommand{\DataTypeTok}[1]{\textcolor[rgb]{0.56,0.13,0.00}{{#1}}}
    \newcommand{\DecValTok}[1]{\textcolor[rgb]{0.25,0.63,0.44}{{#1}}}
    \newcommand{\BaseNTok}[1]{\textcolor[rgb]{0.25,0.63,0.44}{{#1}}}
    \newcommand{\FloatTok}[1]{\textcolor[rgb]{0.25,0.63,0.44}{{#1}}}
    \newcommand{\CharTok}[1]{\textcolor[rgb]{0.25,0.44,0.63}{{#1}}}
    \newcommand{\StringTok}[1]{\textcolor[rgb]{0.25,0.44,0.63}{{#1}}}
    \newcommand{\CommentTok}[1]{\textcolor[rgb]{0.38,0.63,0.69}{\textit{{#1}}}}
    \newcommand{\OtherTok}[1]{\textcolor[rgb]{0.00,0.44,0.13}{{#1}}}
    \newcommand{\AlertTok}[1]{\textcolor[rgb]{1.00,0.00,0.00}{\textbf{{#1}}}}
    \newcommand{\FunctionTok}[1]{\textcolor[rgb]{0.02,0.16,0.49}{{#1}}}
    \newcommand{\RegionMarkerTok}[1]{{#1}}
    \newcommand{\ErrorTok}[1]{\textcolor[rgb]{1.00,0.00,0.00}{\textbf{{#1}}}}
    \newcommand{\NormalTok}[1]{{#1}}
    
    % Additional commands for more recent versions of Pandoc
    \newcommand{\ConstantTok}[1]{\textcolor[rgb]{0.53,0.00,0.00}{{#1}}}
    \newcommand{\SpecialCharTok}[1]{\textcolor[rgb]{0.25,0.44,0.63}{{#1}}}
    \newcommand{\VerbatimStringTok}[1]{\textcolor[rgb]{0.25,0.44,0.63}{{#1}}}
    \newcommand{\SpecialStringTok}[1]{\textcolor[rgb]{0.73,0.40,0.53}{{#1}}}
    \newcommand{\ImportTok}[1]{{#1}}
    \newcommand{\DocumentationTok}[1]{\textcolor[rgb]{0.73,0.13,0.13}{\textit{{#1}}}}
    \newcommand{\AnnotationTok}[1]{\textcolor[rgb]{0.38,0.63,0.69}{\textbf{\textit{{#1}}}}}
    \newcommand{\CommentVarTok}[1]{\textcolor[rgb]{0.38,0.63,0.69}{\textbf{\textit{{#1}}}}}
    \newcommand{\VariableTok}[1]{\textcolor[rgb]{0.10,0.09,0.49}{{#1}}}
    \newcommand{\ControlFlowTok}[1]{\textcolor[rgb]{0.00,0.44,0.13}{\textbf{{#1}}}}
    \newcommand{\OperatorTok}[1]{\textcolor[rgb]{0.40,0.40,0.40}{{#1}}}
    \newcommand{\BuiltInTok}[1]{{#1}}
    \newcommand{\ExtensionTok}[1]{{#1}}
    \newcommand{\PreprocessorTok}[1]{\textcolor[rgb]{0.74,0.48,0.00}{{#1}}}
    \newcommand{\AttributeTok}[1]{\textcolor[rgb]{0.49,0.56,0.16}{{#1}}}
    \newcommand{\InformationTok}[1]{\textcolor[rgb]{0.38,0.63,0.69}{\textbf{\textit{{#1}}}}}
    \newcommand{\WarningTok}[1]{\textcolor[rgb]{0.38,0.63,0.69}{\textbf{\textit{{#1}}}}}
    
    
    
    \def\br{\hspace*{\fill} \\* }
    % Math Jax compatability definitions
    \def\gt{>}
    \def\lt{<}
    % Document parameters
    \title{Evaluación número 1}
    \author{Barajas Ibarria Juan Pedro}
    \date{A 08 de Marzo del 2018}
    
    
\makeatletter
\def\PY@reset{\let\PY@it=\relax \let\PY@bf=\relax%
    \let\PY@ul=\relax \let\PY@tc=\relax%
    \let\PY@bc=\relax \let\PY@ff=\relax}
\def\PY@tok#1{\csname PY@tok@#1\endcsname}
\def\PY@toks#1+{\ifx\relax#1\empty\else%
    \PY@tok{#1}\expandafter\PY@toks\fi}
\def\PY@do#1{\PY@bc{\PY@tc{\PY@ul{%
    \PY@it{\PY@bf{\PY@ff{#1}}}}}}}
\def\PY#1#2{\PY@reset\PY@toks#1+\relax+\PY@do{#2}}

\expandafter\def\csname PY@tok@w\endcsname{\def\PY@tc##1{\textcolor[rgb]{0.73,0.73,0.73}{##1}}}
\expandafter\def\csname PY@tok@c\endcsname{\let\PY@it=\textit\def\PY@tc##1{\textcolor[rgb]{0.25,0.50,0.50}{##1}}}
\expandafter\def\csname PY@tok@cp\endcsname{\def\PY@tc##1{\textcolor[rgb]{0.74,0.48,0.00}{##1}}}
\expandafter\def\csname PY@tok@k\endcsname{\let\PY@bf=\textbf\def\PY@tc##1{\textcolor[rgb]{0.00,0.50,0.00}{##1}}}
\expandafter\def\csname PY@tok@kp\endcsname{\def\PY@tc##1{\textcolor[rgb]{0.00,0.50,0.00}{##1}}}
\expandafter\def\csname PY@tok@kt\endcsname{\def\PY@tc##1{\textcolor[rgb]{0.69,0.00,0.25}{##1}}}
\expandafter\def\csname PY@tok@o\endcsname{\def\PY@tc##1{\textcolor[rgb]{0.40,0.40,0.40}{##1}}}
\expandafter\def\csname PY@tok@ow\endcsname{\let\PY@bf=\textbf\def\PY@tc##1{\textcolor[rgb]{0.67,0.13,1.00}{##1}}}
\expandafter\def\csname PY@tok@nb\endcsname{\def\PY@tc##1{\textcolor[rgb]{0.00,0.50,0.00}{##1}}}
\expandafter\def\csname PY@tok@nf\endcsname{\def\PY@tc##1{\textcolor[rgb]{0.00,0.00,1.00}{##1}}}
\expandafter\def\csname PY@tok@nc\endcsname{\let\PY@bf=\textbf\def\PY@tc##1{\textcolor[rgb]{0.00,0.00,1.00}{##1}}}
\expandafter\def\csname PY@tok@nn\endcsname{\let\PY@bf=\textbf\def\PY@tc##1{\textcolor[rgb]{0.00,0.00,1.00}{##1}}}
\expandafter\def\csname PY@tok@ne\endcsname{\let\PY@bf=\textbf\def\PY@tc##1{\textcolor[rgb]{0.82,0.25,0.23}{##1}}}
\expandafter\def\csname PY@tok@nv\endcsname{\def\PY@tc##1{\textcolor[rgb]{0.10,0.09,0.49}{##1}}}
\expandafter\def\csname PY@tok@no\endcsname{\def\PY@tc##1{\textcolor[rgb]{0.53,0.00,0.00}{##1}}}
\expandafter\def\csname PY@tok@nl\endcsname{\def\PY@tc##1{\textcolor[rgb]{0.63,0.63,0.00}{##1}}}
\expandafter\def\csname PY@tok@ni\endcsname{\let\PY@bf=\textbf\def\PY@tc##1{\textcolor[rgb]{0.60,0.60,0.60}{##1}}}
\expandafter\def\csname PY@tok@na\endcsname{\def\PY@tc##1{\textcolor[rgb]{0.49,0.56,0.16}{##1}}}
\expandafter\def\csname PY@tok@nt\endcsname{\let\PY@bf=\textbf\def\PY@tc##1{\textcolor[rgb]{0.00,0.50,0.00}{##1}}}
\expandafter\def\csname PY@tok@nd\endcsname{\def\PY@tc##1{\textcolor[rgb]{0.67,0.13,1.00}{##1}}}
\expandafter\def\csname PY@tok@s\endcsname{\def\PY@tc##1{\textcolor[rgb]{0.73,0.13,0.13}{##1}}}
\expandafter\def\csname PY@tok@sd\endcsname{\let\PY@it=\textit\def\PY@tc##1{\textcolor[rgb]{0.73,0.13,0.13}{##1}}}
\expandafter\def\csname PY@tok@si\endcsname{\let\PY@bf=\textbf\def\PY@tc##1{\textcolor[rgb]{0.73,0.40,0.53}{##1}}}
\expandafter\def\csname PY@tok@se\endcsname{\let\PY@bf=\textbf\def\PY@tc##1{\textcolor[rgb]{0.73,0.40,0.13}{##1}}}
\expandafter\def\csname PY@tok@sr\endcsname{\def\PY@tc##1{\textcolor[rgb]{0.73,0.40,0.53}{##1}}}
\expandafter\def\csname PY@tok@ss\endcsname{\def\PY@tc##1{\textcolor[rgb]{0.10,0.09,0.49}{##1}}}
\expandafter\def\csname PY@tok@sx\endcsname{\def\PY@tc##1{\textcolor[rgb]{0.00,0.50,0.00}{##1}}}
\expandafter\def\csname PY@tok@m\endcsname{\def\PY@tc##1{\textcolor[rgb]{0.40,0.40,0.40}{##1}}}
\expandafter\def\csname PY@tok@gh\endcsname{\let\PY@bf=\textbf\def\PY@tc##1{\textcolor[rgb]{0.00,0.00,0.50}{##1}}}
\expandafter\def\csname PY@tok@gu\endcsname{\let\PY@bf=\textbf\def\PY@tc##1{\textcolor[rgb]{0.50,0.00,0.50}{##1}}}
\expandafter\def\csname PY@tok@gd\endcsname{\def\PY@tc##1{\textcolor[rgb]{0.63,0.00,0.00}{##1}}}
\expandafter\def\csname PY@tok@gi\endcsname{\def\PY@tc##1{\textcolor[rgb]{0.00,0.63,0.00}{##1}}}
\expandafter\def\csname PY@tok@gr\endcsname{\def\PY@tc##1{\textcolor[rgb]{1.00,0.00,0.00}{##1}}}
\expandafter\def\csname PY@tok@ge\endcsname{\let\PY@it=\textit}
\expandafter\def\csname PY@tok@gs\endcsname{\let\PY@bf=\textbf}
\expandafter\def\csname PY@tok@gp\endcsname{\let\PY@bf=\textbf\def\PY@tc##1{\textcolor[rgb]{0.00,0.00,0.50}{##1}}}
\expandafter\def\csname PY@tok@go\endcsname{\def\PY@tc##1{\textcolor[rgb]{0.53,0.53,0.53}{##1}}}
\expandafter\def\csname PY@tok@gt\endcsname{\def\PY@tc##1{\textcolor[rgb]{0.00,0.27,0.87}{##1}}}
\expandafter\def\csname PY@tok@err\endcsname{\def\PY@bc##1{\setlength{\fboxsep}{0pt}\fcolorbox[rgb]{1.00,0.00,0.00}{1,1,1}{\strut ##1}}}
\expandafter\def\csname PY@tok@kc\endcsname{\let\PY@bf=\textbf\def\PY@tc##1{\textcolor[rgb]{0.00,0.50,0.00}{##1}}}
\expandafter\def\csname PY@tok@kd\endcsname{\let\PY@bf=\textbf\def\PY@tc##1{\textcolor[rgb]{0.00,0.50,0.00}{##1}}}
\expandafter\def\csname PY@tok@kn\endcsname{\let\PY@bf=\textbf\def\PY@tc##1{\textcolor[rgb]{0.00,0.50,0.00}{##1}}}
\expandafter\def\csname PY@tok@kr\endcsname{\let\PY@bf=\textbf\def\PY@tc##1{\textcolor[rgb]{0.00,0.50,0.00}{##1}}}
\expandafter\def\csname PY@tok@bp\endcsname{\def\PY@tc##1{\textcolor[rgb]{0.00,0.50,0.00}{##1}}}
\expandafter\def\csname PY@tok@fm\endcsname{\def\PY@tc##1{\textcolor[rgb]{0.00,0.00,1.00}{##1}}}
\expandafter\def\csname PY@tok@vc\endcsname{\def\PY@tc##1{\textcolor[rgb]{0.10,0.09,0.49}{##1}}}
\expandafter\def\csname PY@tok@vg\endcsname{\def\PY@tc##1{\textcolor[rgb]{0.10,0.09,0.49}{##1}}}
\expandafter\def\csname PY@tok@vi\endcsname{\def\PY@tc##1{\textcolor[rgb]{0.10,0.09,0.49}{##1}}}
\expandafter\def\csname PY@tok@vm\endcsname{\def\PY@tc##1{\textcolor[rgb]{0.10,0.09,0.49}{##1}}}
\expandafter\def\csname PY@tok@sa\endcsname{\def\PY@tc##1{\textcolor[rgb]{0.73,0.13,0.13}{##1}}}
\expandafter\def\csname PY@tok@sb\endcsname{\def\PY@tc##1{\textcolor[rgb]{0.73,0.13,0.13}{##1}}}
\expandafter\def\csname PY@tok@sc\endcsname{\def\PY@tc##1{\textcolor[rgb]{0.73,0.13,0.13}{##1}}}
\expandafter\def\csname PY@tok@dl\endcsname{\def\PY@tc##1{\textcolor[rgb]{0.73,0.13,0.13}{##1}}}
\expandafter\def\csname PY@tok@s2\endcsname{\def\PY@tc##1{\textcolor[rgb]{0.73,0.13,0.13}{##1}}}
\expandafter\def\csname PY@tok@sh\endcsname{\def\PY@tc##1{\textcolor[rgb]{0.73,0.13,0.13}{##1}}}
\expandafter\def\csname PY@tok@s1\endcsname{\def\PY@tc##1{\textcolor[rgb]{0.73,0.13,0.13}{##1}}}
\expandafter\def\csname PY@tok@mb\endcsname{\def\PY@tc##1{\textcolor[rgb]{0.40,0.40,0.40}{##1}}}
\expandafter\def\csname PY@tok@mf\endcsname{\def\PY@tc##1{\textcolor[rgb]{0.40,0.40,0.40}{##1}}}
\expandafter\def\csname PY@tok@mh\endcsname{\def\PY@tc##1{\textcolor[rgb]{0.40,0.40,0.40}{##1}}}
\expandafter\def\csname PY@tok@mi\endcsname{\def\PY@tc##1{\textcolor[rgb]{0.40,0.40,0.40}{##1}}}
\expandafter\def\csname PY@tok@il\endcsname{\def\PY@tc##1{\textcolor[rgb]{0.40,0.40,0.40}{##1}}}
\expandafter\def\csname PY@tok@mo\endcsname{\def\PY@tc##1{\textcolor[rgb]{0.40,0.40,0.40}{##1}}}
\expandafter\def\csname PY@tok@ch\endcsname{\let\PY@it=\textit\def\PY@tc##1{\textcolor[rgb]{0.25,0.50,0.50}{##1}}}
\expandafter\def\csname PY@tok@cm\endcsname{\let\PY@it=\textit\def\PY@tc##1{\textcolor[rgb]{0.25,0.50,0.50}{##1}}}
\expandafter\def\csname PY@tok@cpf\endcsname{\let\PY@it=\textit\def\PY@tc##1{\textcolor[rgb]{0.25,0.50,0.50}{##1}}}
\expandafter\def\csname PY@tok@c1\endcsname{\let\PY@it=\textit\def\PY@tc##1{\textcolor[rgb]{0.25,0.50,0.50}{##1}}}
\expandafter\def\csname PY@tok@cs\endcsname{\let\PY@it=\textit\def\PY@tc##1{\textcolor[rgb]{0.25,0.50,0.50}{##1}}}

\def\PYZbs{\char`\\}
\def\PYZus{\char`\_}
\def\PYZob{\char`\{}
\def\PYZcb{\char`\}}
\def\PYZca{\char`\^}
\def\PYZam{\char`\&}
\def\PYZlt{\char`\<}
\def\PYZgt{\char`\>}
\def\PYZsh{\char`\#}
\def\PYZpc{\char`\%}
\def\PYZdl{\char`\$}
\def\PYZhy{\char`\-}
\def\PYZsq{\char`\'}
\def\PYZdq{\char`\"}
\def\PYZti{\char`\~}
% for compatibility with earlier versions
\def\PYZat{@}
\def\PYZlb{[}
\def\PYZrb{]}
\makeatother


    % Exact colors from NB
    \definecolor{incolor}{rgb}{0.0, 0.0, 0.5}
    \definecolor{outcolor}{rgb}{0.545, 0.0, 0.0}



    
    % Prevent overflowing lines due to hard-to-break entities
    \sloppy 
    % Setup hyperref package
    \hypersetup{
      breaklinks=true,  % so long urls are correctly broken across lines
      colorlinks=true,
      urlcolor=urlcolor,
      linkcolor=linkcolor,
      citecolor=citecolor,
      }
    % Slightly bigger margins than the latex defaults
    
    \geometry{verbose,tmargin=1in,bmargin=1in,lmargin=1in,rmargin=1in}
    
    

    \begin{document}
    
    
    \maketitle
    \begin{center}
    \adjustimage{max size={0.4\linewidth}{0.6\paperheight}}{escudo.png}
    \end{center}
	\begin{center}
    \section*{Introducción}
    \end{center}
    Se tiene una estación de monitoreo de variables atmosféricas, CO2, radiación solar, nivel de agua y salinidad en el Manglar El Sargento, en una bahía en la costa frente a la parte norte de la Isla Tiburón.
    
    \begin{center}
    \adjustimage{max size={0.4\linewidth}{0.6\paperheight}}{SargentoMap.png}
    \vspace{3cm}
  
    \end{center}
    \begin{center}
    \adjustimage{max size={0.6\linewidth}{0.9\paperheight}}{SargentoMap2.png}
    \end{center}
   \begin{abstract}
    \subsubsection*{La isla tiburón}
   La isla Tiburón~\cite{Isla:MISC} o isla del Tiburón. es una isla mexicana, la más grande del golfo de California y tiene una extensión de 1208 km cuadrados. Administrativamente, pertenece al estado de Sonora, específicamente al municipio de Hermosillo, y se ubica aproximadamente a la misma latitud que Hermosillo. Está separada del continente por un estrecho canal de sólo tres kilómetros de ancho llamado estrecho del Infiernillo.\par
La isla está deshabitada, a excepción de una instalación militar ubicada en la zona oriental de la isla. Está administrada como una reserva ecológica por el gobierno de los seris, en conjunto con el gobierno Federal. En siglos anteriores, la isla fue habitada por tres grupos ("bandas") de los seris: los Tahejöc comcaac, los heeno comcaac y los xiica Hast ano coii (en una parte). Esta isla es para los seris un sitio sagrado, ya que consideran la isla como la cuna de su pueblo.
	\subsubsection*{Manglar}
   
   El manglar~\cite{Manglar:Misc} es un área biótica o bioma, formado por árboles muy tolerantes a las sales existentes en la zona intermareal cercana a la desembocadura de cursos de agua dulce en latitudes tropicales y subtropicales. Así, entre las áreas con manglares se incluyen estuarios y zonas costeras. Tienen una gran diversidad biológica con alta productividad, encontrándose muchas especies de aves como de peces, crustáceos, moluscos y otras.\par
Su nombre deriva de los árboles que los forman, los mangles, el vocablo mangle (de donde se deriva mangrove en alemán, francés e inglés) proviene de una voz caribe o arahuaca,1 quizá guaraní[cita requerida] y significa árbol retorcido. Normalmente se dan como barrera motivos de desarrollo, la costa ha sufrido una rápida erosión. También sirven de hábitat para numerosas especies y proporcionan una protección natural contra fuertes vientos, olas producidas por huracanes e incluso por maremotos.
   \end{abstract}
   \begin{center}
   \section*{Desarrollo de la actividad}
   \end{center}
     \begin{itemize}
    \item 
 	La estructura de datos esta dada con una por una separación por columna 
   	en el caso de los datos de salinidad tenemos 6 columnas.
    \end{itemize}
    \begin{enumerate}
    \item Al iniciar con la actividad se descargan dos archivos de datos de los meses de octubre y noviembre del 2017 de la estación \textsl{El Sargento} los cuales los cuales fueron tomados en intervalos de 15 minutos los cuales analizaremos.
    \item Descargando los datos pasamos a ejecutar jupyter notebook donde en Pandas cargamos las librerias Numpy, Pyplot de Matplotlib y la biblioteca datetime, continuando cargamos los dos archivos de datos los cuales son:
    \begin{itemize}
    \item Sargento17.csv~\cite{Datos1:Misc}, cargado en la variable dfsar
    \item Sargento17sanidad~\cite{Datos2:Misc}, cargad en la variable dfsal
    \end{itemize}
    Donde agregamos los nombres de las columnas igual que en el archivo original y enseguida pasamos a darle la estructura de datos  bajo los nombres que aparecen los siguientes segmentos de código.
    \begin{Verbatim}[commandchars=\\\{\}]
{\color{incolor}In [{\color{incolor}1}]:} \PY{k+kn}{import} \PY{n+nn}{pandas} \PY{k}{as} \PY{n+nn}{pd}
        \PY{k+kn}{import} \PY{n+nn}{numpy} \PY{k}{as} \PY{n+nn}{np} 
        \PY{k+kn}{from} \PY{n+nn}{datetime} \PY{k}{import} \PY{n}{datetime}
        \PY{k+kn}{import} \PY{n+nn}{matplotlib}\PY{n+nn}{.}\PY{n+nn}{pyplot} \PY{k}{as} \PY{n+nn}{plt}
\end{Verbatim}


    \begin{Verbatim}[commandchars=\\\{\}]
{\color{incolor}In [{\color{incolor}2}]:} \PY{c+c1}{\PYZsh{} Lee un archivo de texto con la función Pandas \PYZdq{}read\PYZus{}csv\PYZdq{}, con elementos separados por mas de }
        \PY{c+c1}{\PYZsh{} un espacio, brincándose 4 renglones del inicio (encabezados)}
        \PY{n}{dfsar} \PY{o}{=} \PY{n}{pd}\PY{o}{.}\PY{n}{read\PYZus{}csv}\PY{p}{(}\PY{l+s+s1}{\PYZsq{}}\PY{l+s+s1}{sargento17.csv}\PY{l+s+s1}{\PYZsq{}}\PY{p}{,} \PY{n}{skiprows}\PY{o}{=}\PY{l+m+mi}{2}\PY{p}{,} \PY{n}{sep}\PY{o}{=}\PY{l+s+s1}{\PYZsq{}}\PY{l+s+s1}{,}\PY{l+s+s1}{\PYZsq{}}\PY{p}{,}\PY{n}{names}\PY{o}{=}\PY{p}{[}\PY{l+s+s1}{\PYZsq{}}\PY{l+s+s1}{\PYZsh{}}\PY{l+s+s1}{\PYZsq{}}\PY{p}{,}\PY{l+s+s1}{\PYZsq{}}\PY{l+s+s1}{Datetime}
        \PY{l+s+s1}{\PYZsq{}}\PY{p}{,}\PY{l+s+s1}{\PYZsq{}}\PY{l+s+s1}{Abspres}\PY{l+s+s1}{\PYZsq{}}\PY{p}{,}\PY{l+s+s1}{\PYZsq{}}\PY{l+s+s1}{Temp}\PY{l+s+s1}{\PYZsq{}}\PY{p}{,}\PY{l+s+s1}{\PYZsq{}}\PY{l+s+s1}{Waterlevel}\PY{l+s+s1}{\PYZsq{}}\PY{p}{]}\PY{p}{)}
        \PY{n}{dfsal} \PY{o}{=} \PY{n}{pd}\PY{o}{.}\PY{n}{read\PYZus{}csv}\PY{p}{(}\PY{l+s+s1}{\PYZsq{}}\PY{l+s+s1}{sargento17sanidad.csv}\PY{l+s+s1}{\PYZsq{}}\PY{p}{,} \PY{n}{skiprows}\PY{o}{=}\PY{l+m+mi}{2}\PY{p}{,} \PY{n}{sep}\PY{o}{=}\PY{l+s+s1}{\PYZsq{}}\PY{l+s+s1}{,}\PY{l+s+s1}{\PYZsq{}}\PY{p}{,}\PY{n}{names}\PY{o}{=}\PY{p}{[}\PY{l+s+s1}{\PYZsq{}}\PY{l+s+s1}{\PYZsh{}}\PY{l+s+s1}{\PYZsq{}}\PY{p}{,}\PY{l+s+s1}{\PYZsq{}}\PY{l+s+s1}{Datetime}
        \PY{l+s+s1}{\PYZsq{}}\PY{p}{,}\PY{l+s+s1}{\PYZsq{}}\PY{l+s+s1}{CondhR}\PY{l+s+s1}{\PYZsq{}}\PY{p}{,}\PY{l+s+s1}{\PYZsq{}}\PY{l+s+s1}{Temp}\PY{l+s+s1}{\PYZsq{}}\PY{p}{,}\PY{l+s+s1}{\PYZsq{}}\PY{l+s+s1}{SpecCon}\PY{l+s+s1}{\PYZsq{}}\PY{p}{,}\PY{l+s+s1}{\PYZsq{}}\PY{l+s+s1}{Sal}\PY{l+s+s1}{\PYZsq{}}\PY{p}{]}\PY{p}{)}
        \PY{c+c1}{\PYZsh{} \PYZdq{}Shift + Enter\PYZdq{}}
\end{Verbatim}


    \begin{Verbatim}[commandchars=\\\{\}]
{\color{incolor}In [{\color{incolor}4}]:} \PY{n}{dfsarg}\PY{o}{=}\PY{n}{pd}\PY{o}{.}\PY{n}{DataFrame}\PY{p}{(}\PY{n}{dfsar}\PY{p}{)}
        \PY{n}{dfsani}\PY{o}{=}\PY{n}{pd}\PY{o}{.}\PY{n}{DataFrame}\PY{p}{(}\PY{n}{dfsal}\PY{p}{)}
\end{Verbatim}

\item Después de darle la estructura de datos pasamos a convertir la columna de datos datetime a datos temporales  como se muestra en el segmento de código, estos acompañados con el comando \textsl{head()} el cual los muestra de tal manera que cada columna de datos tiene nombre.

    \begin{Verbatim}[commandchars=\\\{\}]
{\color{incolor}In [{\color{incolor}5}]:} \PY{c+c1}{\PYZsh{} Convertir la cadena de caracteres \PYZsq{}Date\PYZsq{} en variable temporal \PYZsq{}NDate\PYZsq{}}
        \PY{n}{dfsani}\PY{p}{[}\PY{l+s+s1}{\PYZsq{}}\PY{l+s+s1}{Ndate}\PY{l+s+s1}{\PYZsq{}}\PY{p}{]} \PY{o}{=} \PY{n}{pd}\PY{o}{.}\PY{n}{to\PYZus{}datetime}\PY{p}{(}\PY{n}{dfsani}\PY{p}{[}\PY{l+s+s1}{\PYZsq{}}\PY{l+s+s1}{Datetime}\PY{l+s+s1}{\PYZsq{}}\PY{p}{]}\PY{p}{,} \PY{n+nb}{format}\PY{o}{=}\PY{l+s+s1}{\PYZsq{}}\PY{l+s+s1}{\PYZpc{}}\PY{l+s+s1}{m/}\PY{l+s+si}{\PYZpc{}d}\PY{l+s+s1}{/}\PY{l+s+s1}{\PYZpc{}}\PY{l+s+s1}{Y }\PY{l+s+s1}{\PYZpc{}}\PY{l+s+s1}{H:}\PY{l+s+s1}{\PYZpc{}}\PY{l+s+s1}{M:}\PY{l+s+s1}{\PYZpc{}}\PY{l+s+s1}{S}\PY{l+s+s1}{\PYZsq{}}\PY{p}{)}
        \PY{n}{dfsani}\PY{p}{[}\PY{l+s+s1}{\PYZsq{}}\PY{l+s+s1}{month}\PY{l+s+s1}{\PYZsq{}}\PY{p}{]} \PY{o}{=} \PY{n}{dfsani}\PY{p}{[}\PY{l+s+s1}{\PYZsq{}}\PY{l+s+s1}{Ndate}\PY{l+s+s1}{\PYZsq{}}\PY{p}{]}\PY{o}{.}\PY{n}{dt}\PY{o}{.}\PY{n}{month}
        \PY{n}{dfsani}\PY{o}{.}\PY{n}{head}\PY{p}{(}\PY{p}{)}
\end{Verbatim}


\begin{Verbatim}[commandchars=\\\{\}]
{\color{outcolor}Out[{\color{outcolor}5}]:}    \#             Datetime   CondhR   Temp  SpecCon      Sal  \textbackslash{}
        0  1  10/26/2017 12:45:00  54525.5  25.21  54301.2  35.9195   
        1  2  10/26/2017 13:00:00  54525.5  24.91  54622.1  36.1588   
        2  3  10/26/2017 13:15:00  54525.5  24.82  54719.0  36.2311   
        3  4  10/26/2017 13:30:00  54525.5  24.76  54783.8  36.2794   
        4  5  10/26/2017 13:45:00  54525.5  24.75  54794.6  36.2875   
        
                        Ndate  month  
        0 2017-10-26 12:45:00     10  
        1 2017-10-26 13:00:00     10  
        2 2017-10-26 13:15:00     10  
        3 2017-10-26 13:30:00     10  
        4 2017-10-26 13:45:00     10  
\end{Verbatim}
            
    \begin{Verbatim}[commandchars=\\\{\}]
{\color{incolor}In [{\color{incolor}6}]:} \PY{c+c1}{\PYZsh{} Convertir la cadena de caracteres \PYZsq{}Date\PYZsq{} en variable temporal \PYZsq{}NDate\PYZsq{}}
        \PY{n}{dfsarg}\PY{p}{[}\PY{l+s+s1}{\PYZsq{}}\PY{l+s+s1}{Ndate}\PY{l+s+s1}{\PYZsq{}}\PY{p}{]} \PY{o}{=} \PY{n}{pd}\PY{o}{.}\PY{n}{to\PYZus{}datetime}\PY{p}{(}\PY{n}{dfsarg}\PY{p}{[}\PY{l+s+s1}{\PYZsq{}}\PY{l+s+s1}{Datetime}\PY{l+s+s1}{\PYZsq{}}\PY{p}{]}\PY{p}{,} \PY{n+nb}{format}\PY{o}{=}\PY{l+s+s1}{\PYZsq{}}\PY{l+s+s1}{\PYZpc{}}\PY{l+s+s1}{m/}\PY{l+s+si}{\PYZpc{}d}\PY{l+s+s1}{/}\PY{l+s+s1}{\PYZpc{}}\PY{l+s+s1}{Y }\PY{l+s+s1}{\PYZpc{}}\PY{l+s+s1}{H:}\PY{l+s+s1}{\PYZpc{}}\PY{l+s+s1}{M:}\PY{l+s+s1}{\PYZpc{}}\PY{l+s+s1}{S}\PY{l+s+s1}{\PYZsq{}}\PY{p}{)}
        \PY{n}{dfsarg}\PY{p}{[}\PY{l+s+s1}{\PYZsq{}}\PY{l+s+s1}{month}\PY{l+s+s1}{\PYZsq{}}\PY{p}{]} \PY{o}{=} \PY{n}{dfsarg}\PY{p}{[}\PY{l+s+s1}{\PYZsq{}}\PY{l+s+s1}{Ndate}\PY{l+s+s1}{\PYZsq{}}\PY{p}{]}\PY{o}{.}\PY{n}{dt}\PY{o}{.}\PY{n}{month}
        \PY{n}{dfsarg}\PY{o}{.}\PY{n}{head}\PY{p}{(}\PY{p}{)}
\end{Verbatim}


\begin{Verbatim}[commandchars=\\\{\}]
{\color{outcolor}Out[{\color{outcolor}6}]:}    \#             Datetime  Abspres    Temp  Waterlevel               Ndate  \textbackslash{}
        0  1  10/26/2017 13:00:00  105.612  24.448      -0.150 2017-10-26 13:00:00   
        1  2  10/26/2017 13:15:00  105.513  24.351      -0.160 2017-10-26 13:15:00   
        2  3  10/26/2017 13:30:00  105.433  24.351      -0.168 2017-10-26 13:30:00   
        3  4  10/26/2017 13:45:00  105.385  24.351      -0.173 2017-10-26 13:45:00   
        4  5  10/26/2017 14:00:00  105.321  24.351      -0.179 2017-10-26 14:00:00   
        
           month  
        0     10  
        1     10  
        2     10  
        3     10  
        4     10  
\end{Verbatim}
            \item Al terminar de convertir todos los datos necesarios continuamos con las gráficas, donde con ayuda de seaborn~\cite{Regresión:Misc} se realizaron gráficas de caja para: 
         \begin{enumerate}
         \item a) Nivel del mar (Metros)
         \item b) Salinidad (PPM)
         \item c)Temperatura de agua (grados centigrados )
         \end{enumerate}
         Respectivamente como se muestran a continuación con el segmento de código utilizado seguido de la imagen de la gráfica de caja.\par
         También después de ellas presentamos con el comando \textsl{.describe()} sus datos estadísticos principales para poder interpretar los diagramas de caja.\par
         \vspace{1.5 cm}
         \huge{a)}\normalsize\begin{Verbatim}[commandchars=\\\{\}]
{\color{incolor}In [{\color{incolor}7}]:} \PY{c+c1}{\PYZsh{} graficar Boxplots por mes}
        \PY{c+c1}{\PYZsh{} Biblioteca Seaborn}
        \PY{k+kn}{import} \PY{n+nn}{seaborn} \PY{k}{as} \PY{n+nn}{sns}
        \PY{k+kn}{import} \PY{n+nn}{matplotlib}\PY{n+nn}{.}\PY{n+nn}{pyplot} \PY{k}{as} \PY{n+nn}{plt}
        \PY{n}{ax} \PY{o}{=} \PY{n}{sns}\PY{o}{.}\PY{n}{boxplot}\PY{p}{(}\PY{n}{x}\PY{o}{=}\PY{l+s+s2}{\PYZdq{}}\PY{l+s+s2}{month}\PY{l+s+s2}{\PYZdq{}}\PY{p}{,} \PY{n}{y}\PY{o}{=}\PY{l+s+s2}{\PYZdq{}}\PY{l+s+s2}{Waterlevel}\PY{l+s+s2}{\PYZdq{}}\PY{p}{,} \PY{n}{data}\PY{o}{=}\PY{n}{dfsarg}\PY{p}{)}
        \PY{n}{plt}\PY{o}{.}\PY{n}{show}\PY{p}{(}\PY{p}{)}
\end{Verbatim}


    \begin{center}
    \adjustimage{max size={0.9\linewidth}{0.9\paperheight}}{output_5_0.png}
    \end{center}
    { \hspace*{\fill} \\}
     \huge{b)}\normalsize
    \begin{Verbatim}[commandchars=\\\{\}]
{\color{incolor}In [{\color{incolor}9}]:} \PY{c+c1}{\PYZsh{} graficar Boxplots por mes}
        \PY{c+c1}{\PYZsh{} Biblioteca Seaborn}
        \PY{k+kn}{import} \PY{n+nn}{seaborn} \PY{k}{as} \PY{n+nn}{sns}
        \PY{k+kn}{import} \PY{n+nn}{matplotlib}\PY{n+nn}{.}\PY{n+nn}{pyplot} \PY{k}{as} \PY{n+nn}{plt}
        \PY{n}{ax} \PY{o}{=} \PY{n}{sns}\PY{o}{.}\PY{n}{boxplot}\PY{p}{(}\PY{n}{x}\PY{o}{=}\PY{l+s+s2}{\PYZdq{}}\PY{l+s+s2}{month}\PY{l+s+s2}{\PYZdq{}}\PY{p}{,} \PY{n}{y}\PY{o}{=}\PY{l+s+s2}{\PYZdq{}}\PY{l+s+s2}{Sal}\PY{l+s+s2}{\PYZdq{}}\PY{p}{,} \PY{n}{data}\PY{o}{=}\PY{n}{dfsani}\PY{p}{)}
        \PY{n}{plt}\PY{o}{.}\PY{n}{show}\PY{p}{(}\PY{p}{)}
\end{Verbatim}


    \begin{center}
    \adjustimage{max size={0.9\linewidth}{0.9\paperheight}}{output_6_0.png}
    \end{center}
    { \hspace*{\fill} \\}
     \huge{c)}\normalsize
    \begin{Verbatim}[commandchars=\\\{\}]
{\color{incolor}In [{\color{incolor}11}]:} \PY{c+c1}{\PYZsh{} graficar Boxplots por mes}
         \PY{c+c1}{\PYZsh{} Biblioteca Seaborn}
         \PY{k+kn}{import} \PY{n+nn}{seaborn} \PY{k}{as} \PY{n+nn}{sns}
         \PY{k+kn}{import} \PY{n+nn}{matplotlib}\PY{n+nn}{.}\PY{n+nn}{pyplot} \PY{k}{as} \PY{n+nn}{plt}
         \PY{n}{ax} \PY{o}{=} \PY{n}{sns}\PY{o}{.}\PY{n}{boxplot}\PY{p}{(}\PY{n}{x}\PY{o}{=}\PY{l+s+s2}{\PYZdq{}}\PY{l+s+s2}{month}\PY{l+s+s2}{\PYZdq{}}\PY{p}{,} \PY{n}{y}\PY{o}{=}\PY{l+s+s2}{\PYZdq{}}\PY{l+s+s2}{Temp}\PY{l+s+s2}{\PYZdq{}}\PY{p}{,} \PY{n}{data}\PY{o}{=}\PY{n}{dfsani}\PY{p}{)}
         \PY{n}{plt}\PY{o}{.}\PY{n}{show}\PY{p}{(}\PY{p}{)}
\end{Verbatim}


    \begin{center}
    \adjustimage{max size={0.9\linewidth}{0.9\paperheight}}{output_7_0.png}
    \end{center}
    { \hspace*{\fill} \\}
    
    \begin{Verbatim}[commandchars=\\\{\}]
{\color{incolor}In [{\color{incolor}12}]:} \PY{n}{dfsarg}\PY{o}{.}\PY{n}{describe}\PY{p}{(}\PY{p}{)}
\end{Verbatim}


\begin{Verbatim}[commandchars=\\\{\}]
{\color{outcolor}Out[{\color{outcolor}12}]:}                 \#      Abspres         Temp   Waterlevel        month
         count  2395.00000  2395.000000  2395.000000  2395.000000  2395.000000
         mean   1198.00000   107.429820    23.120315     0.030845    10.781211
         std     691.52127     2.371366     0.564123     0.235926     0.413512
         min       1.00000   104.229000    21.760000    -0.288000    10.000000
         25\%     599.50000   106.407000    22.525000    -0.071000    11.000000
         50\%    1198.00000   106.764000    23.388000    -0.035000    11.000000
         75\%    1796.50000   107.303000    23.484000     0.018500    11.000000
         max    2395.00000   118.641000    24.448000     1.146000    11.000000
\end{Verbatim}
            
    \begin{Verbatim}[commandchars=\\\{\}]
{\color{incolor}In [{\color{incolor}13}]:} \PY{n}{dfsani}\PY{o}{.}\PY{n}{describe}\PY{p}{(}\PY{p}{)}
\end{Verbatim}


\begin{Verbatim}[commandchars=\\\{\}]
{\color{outcolor}Out[{\color{outcolor}13}]:}                 \#        CondhR         Temp       SpecCon          Sal  \textbackslash{}
         count  2395.00000   2395.000000  2395.000000   2395.000000  2395.000000   
         mean   1198.00000  54524.973027    23.317436  56385.960835    37.479086   
         std     691.52127     11.874193     0.548286    620.837037     0.465969   
         min       1.00000  54105.700000    21.490000  54301.200000    35.919500   
         25\%     599.50000  54525.500000    22.730000  55949.700000    37.151400   
         50\%    1198.00000  54525.500000    23.490000  56185.600000    37.328300   
         75\%    1796.50000  54525.500000    23.700000  57053.700000    37.980300   
         max    2395.00000  54525.500000    25.210000  58398.700000    38.994200   
         
                      month  
         count  2395.000000  
         mean     10.780793  
         std       0.413795  
         min      10.000000  
         25\%      11.000000  
         50\%      11.000000  
         75\%      11.000000  
         max      11.000000  
\end{Verbatim}
            Podemos apreciar en las gráficas que, en a:
            \begin{itemize}
            \item Los niveles de agua crecieron muy poco en noviembre, se puede notar que los valores máximo y mínimo son 1.146m y 0.288 m respectivamente. 
            \item En octubre la media se mantuvo muy centrada en el primer y segundo cuarti y en noviembre de igual forma se concentran en los primeros cuartiles pero la media es menor aquí.
            \end{itemize}
            En b:
              \begin{itemize}
            \item La salinidad media de estos dos meses  es 37.4790 ppm donde la distribución encuentra la media alejada del centro de la concentración de la caja, lo cual nos dice que no vario mucho fuera de ese foco.
            \item También vemos que la máxima y mínima son 38.994ppm y 35.919 ppm respectivamente.
            
            \end{itemize}
            En c:
            \begin{itemize}
            \item La La temperatura media fue 23.12.. grados centigrados donde la distribución de los dos cuartiles centrales están por arriba de 23 grados y en noviembre tiene una disminución con la distribución mas amplia. 
            \item Las temperaturas máxima y mínima son 24.44 grados centigrados y 21.76 grados centigrados respectivamente.
            \end{itemize}
            
          \item Después de sacar los datos estadísticos principales pasamos a generar gráficas de  de regresión lineal con distribuciones marginales, donde cada gráfica contiene el nombre de sus ejes para interpretar su relación lineal.
          
    \begin{Verbatim}[commandchars=\\\{\}]
{\color{incolor}In [{\color{incolor}27}]:} \PY{c+c1}{\PYZsh{} Gráfica de Nivel de mar\PYZhy{}Salinidad}
         \PY{n}{sns}\PY{o}{.}\PY{n}{set}\PY{p}{(}\PY{n}{style}\PY{o}{=}\PY{l+s+s2}{\PYZdq{}}\PY{l+s+s2}{darkgrid}\PY{l+s+s2}{\PYZdq{}}\PY{p}{,} \PY{n}{color\PYZus{}codes}\PY{o}{=}\PY{k+kc}{True}\PY{p}{)}
         \PY{n}{dfjuntos}\PY{o}{=}\PY{n}{pd}\PY{o}{.}\PY{n}{concat}\PY{p}{(}\PY{p}{[}\PY{n}{dfsarg}\PY{p}{,} \PY{n}{dfsani}\PY{p}{]}\PY{p}{,} \PY{n}{axis}\PY{o}{=}\PY{l+m+mi}{1}\PY{p}{,} \PY{n}{join\PYZus{}axes}\PY{o}{=}\PY{p}{[}\PY{n}{dfsani}\PY{o}{.}\PY{n}{index}\PY{p}{]}\PY{p}{)}
         \PY{n}{g} \PY{o}{=} \PY{n}{sns}\PY{o}{.}\PY{n}{jointplot}\PY{p}{(}\PY{l+s+s2}{\PYZdq{}}\PY{l+s+s2}{Sal}\PY{l+s+s2}{\PYZdq{}}\PY{p}{,} \PY{l+s+s2}{\PYZdq{}}\PY{l+s+s2}{Waterlevel}\PY{l+s+s2}{\PYZdq{}}\PY{p}{,} \PY{n}{data}\PY{o}{=}\PY{n}{dfjuntos}\PY{p}{,}\PY{n}{kind}\PY{o}{=}\PY{l+s+s2}{\PYZdq{}}\PY{l+s+s2}{reg}\PY{l+s+s2}{\PYZdq{}}\PY{p}{,} \PY{n}{color}\PY{o}{=}\PY{l+s+s2}{\PYZdq{}}\PY{l+s+s2}{c}\PY{l+s+s2}{\PYZdq{}}\PY{p}{)}
         \PY{n}{plt}\PY{o}{.}\PY{n}{show}\PY{p}{(}\PY{n}{g}\PY{p}{)}
\end{Verbatim}


    \begin{center}
    \adjustimage{max size={0.9\linewidth}{0.9\paperheight}}{output_10_0.png}
    \end{center}
    { \hspace*{\fill} \\}
      En la primer gráfica de nivel del agua contra salinidad vemos que hay una relación lineal minima o casi nula donde pearsonr=0.13 lo cual nos confirma que los datos tienen muy pero muy poca relación lineal.
    \begin{Verbatim}[commandchars=\\\{\}]
{\color{incolor}In [{\color{incolor}31}]:} \PY{c+c1}{\PYZsh{}Nivel del agua\PYZhy{}temperatura}
         \PY{k+kn}{import} \PY{n+nn}{seaborn} \PY{k}{as} \PY{n+nn}{sns}
         \PY{n}{sns}\PY{o}{.}\PY{n}{set}\PY{p}{(}\PY{n}{style}\PY{o}{=}\PY{l+s+s2}{\PYZdq{}}\PY{l+s+s2}{darkgrid}\PY{l+s+s2}{\PYZdq{}}\PY{p}{,} \PY{n}{color\PYZus{}codes}\PY{o}{=}\PY{k+kc}{True}\PY{p}{)}
         
         \PY{n}{g} \PY{o}{=} \PY{n}{sns}\PY{o}{.}\PY{n}{jointplot}\PY{p}{(}\PY{l+s+s2}{\PYZdq{}}\PY{l+s+s2}{Waterlevel}\PY{l+s+s2}{\PYZdq{}}\PY{p}{,}\PY{l+s+s2}{\PYZdq{}}\PY{l+s+s2}{Temp}\PY{l+s+s2}{\PYZdq{}}\PY{p}{,} \PY{n}{data}\PY{o}{=}\PY{n}{dfsarg}\PY{p}{,} \PY{n}{kind}\PY{o}{=}\PY{l+s+s2}{\PYZdq{}}\PY{l+s+s2}{reg}\PY{l+s+s2}{\PYZdq{}}\PY{p}{,}
                            \PY{n}{color}\PY{o}{=}\PY{l+s+s2}{\PYZdq{}}\PY{l+s+s2}{r}\PY{l+s+s2}{\PYZdq{}}\PY{p}{,} \PY{n}{size}\PY{o}{=}\PY{l+m+mi}{7}\PY{p}{)}
         \PY{n}{plt}\PY{o}{.}\PY{n}{show}\PY{p}{(}\PY{n}{g}\PY{p}{)}
\end{Verbatim}


    \begin{center}
    \adjustimage{max size={0.9\linewidth}{0.9\paperheight}}{output_11_0.png}
    \end{center}
    { \hspace*{\fill} \\}
  	En la segunda gráfica de temperatura contra nivel del agua vemos vemos que la relación lineal de nuevo tiene muy poca relación, esto nos dice que los datos no tienen mucho que ver con los otros, para confirmar esto vemos que pearsonr=0.18.
    \begin{Verbatim}[commandchars=\\\{\}]
{\color{incolor}In [{\color{incolor}32}]:} \PY{c+c1}{\PYZsh{}Salinidad\PYZhy{}Temperatura}
         \PY{k+kn}{import} \PY{n+nn}{seaborn} \PY{k}{as} \PY{n+nn}{sns}
         \PY{n}{sns}\PY{o}{.}\PY{n}{set}\PY{p}{(}\PY{n}{style}\PY{o}{=}\PY{l+s+s2}{\PYZdq{}}\PY{l+s+s2}{darkgrid}\PY{l+s+s2}{\PYZdq{}}\PY{p}{,} \PY{n}{color\PYZus{}codes}\PY{o}{=}\PY{k+kc}{True}\PY{p}{)}
         
         \PY{n}{g} \PY{o}{=} \PY{n}{sns}\PY{o}{.}\PY{n}{jointplot}\PY{p}{(}\PY{l+s+s2}{\PYZdq{}}\PY{l+s+s2}{Sal}\PY{l+s+s2}{\PYZdq{}}\PY{p}{,}\PY{l+s+s2}{\PYZdq{}}\PY{l+s+s2}{Temp}\PY{l+s+s2}{\PYZdq{}}\PY{p}{,} \PY{n}{data}\PY{o}{=}\PY{n}{dfsani}\PY{p}{,} \PY{n}{kind}\PY{o}{=}\PY{l+s+s2}{\PYZdq{}}\PY{l+s+s2}{reg}\PY{l+s+s2}{\PYZdq{}}\PY{p}{,}
                            \PY{n}{color}\PY{o}{=}\PY{l+s+s2}{\PYZdq{}}\PY{l+s+s2}{r}\PY{l+s+s2}{\PYZdq{}}\PY{p}{,} \PY{n}{size}\PY{o}{=}\PY{l+m+mi}{7}\PY{p}{)}
         \PY{n}{plt}\PY{o}{.}\PY{n}{show}\PY{p}{(}\PY{n}{g}\PY{p}{)}
\end{Verbatim}


    \begin{center}
    \adjustimage{max size={0.9\linewidth}{0.9\paperheight}}{output_12_0.png}
    \end{center}
    { \hspace*{\fill} \\}
    En la tercer gráfica de temperatura contra salinidad del agua vemos una recta muy bien a ajustada a los datos, lo cual intuitivamente nos muestra la relación directa de estos dos datos, que nos dice que mientras la temperatura descienda la concentración de sal sera mayor, esto lo confirma el dato pearsonr=-1, que efectivamente como se predice, la gráfica muestra una asociación lineal negativa perfecta.
    \item Continuando con el análisis realizamos tres gráficas independientes de las variables: d) Nivel del mar, e) Salinidad y f) Temperatura del Agua, para ver su variabilidad como función del tiempo, las cuales se encuentran en orden respectivo junto con su segmento de código con el que se generaron.\par
    
    \huge{d)}\normalsize \begin{Verbatim}[commandchars=\\\{\}]
{\color{incolor}In [{\color{incolor}42}]:} \PY{n}{plt}\PY{o}{.}\PY{n}{plot\PYZus{}date}\PY{p}{(}\PY{n}{x}\PY{o}{=}\PY{n}{dfsani}\PY{o}{.}\PY{n}{month}\PY{p}{,} \PY{n}{y}\PY{o}{=}\PY{n}{dfsarg}\PY{o}{.}\PY{n}{Waterlevel}\PY{p}{,} \PY{n}{fmt}\PY{o}{=}\PY{l+s+s2}{\PYZdq{}}\PY{l+s+s2}{b\PYZhy{}}\PY{l+s+s2}{\PYZdq{}}\PY{p}{)}
         \PY{n}{plt}\PY{o}{.}\PY{n}{title}\PY{p}{(}\PY{l+s+s2}{\PYZdq{}}\PY{l+s+s2}{Nivel del agua}\PY{l+s+s2}{\PYZdq{}}\PY{p}{)}
         \PY{n}{plt}\PY{o}{.}\PY{n}{ylabel}\PY{p}{(}\PY{l+s+s2}{\PYZdq{}}\PY{l+s+s2}{Metros}\PY{l+s+s2}{\PYZdq{}}\PY{p}{)}
         \PY{n}{plt}\PY{o}{.}\PY{n}{grid}\PY{p}{(}\PY{k+kc}{True}\PY{p}{)}
         \PY{n}{plt}\PY{o}{.}\PY{n}{show}\PY{p}{(}\PY{p}{)}
\end{Verbatim}


    \begin{center}
    \adjustimage{max size={0.9\linewidth}{0.9\paperheight}}{output_13_0.png}
    \end{center}
    { \hspace*{\fill} \\}
    En esta gráfica vemos, como se dijo en las gráficas de caja pero de manera mas clara, que en noviembre encontramos un ligero aumento en el nivel del agua conforme avanzaba el tiempo habiendo unos valores un poco extremos pero que igual se encuentra en su mayoría al rededor de la media con una desviación estándar de 0.2358.\par
    \huge{e)}\normalsize \begin{Verbatim}[commandchars=\\\{\}]
{\color{incolor}In [{\color{incolor}44}]:} \PY{n}{plt}\PY{o}{.}\PY{n}{plot\PYZus{}date}\PY{p}{(}\PY{n}{x}\PY{o}{=}\PY{n}{dfsani}\PY{o}{.}\PY{n}{month}\PY{p}{,} \PY{n}{y}\PY{o}{=}\PY{n}{dfsani}\PY{o}{.}\PY{n}{Sal}\PY{p}{,} \PY{n}{fmt}\PY{o}{=}\PY{l+s+s2}{\PYZdq{}}\PY{l+s+s2}{b\PYZhy{}}\PY{l+s+s2}{\PYZdq{}}\PY{p}{)}
         \PY{n}{plt}\PY{o}{.}\PY{n}{title}\PY{p}{(}\PY{l+s+s2}{\PYZdq{}}\PY{l+s+s2}{Salinidad}\PY{l+s+s2}{\PYZdq{}}\PY{p}{)}
         \PY{n}{plt}\PY{o}{.}\PY{n}{ylabel}\PY{p}{(}\PY{l+s+s2}{\PYZdq{}}\PY{l+s+s2}{PPT}\PY{l+s+s2}{\PYZdq{}}\PY{p}{)}
         \PY{n}{plt}\PY{o}{.}\PY{n}{grid}\PY{p}{(}\PY{k+kc}{True}\PY{p}{)}
         \PY{n}{plt}\PY{o}{.}\PY{n}{show}\PY{p}{(}\PY{p}{)}
\end{Verbatim}


    \begin{center}
    \adjustimage{max size={0.9\linewidth}{0.9\paperheight}}{output_14_0.png}
    \end{center}
    { \hspace*{\fill} \\}
    En esta gráfica encontramos también la relación directa con los diagramas e caja donde su comportamiento se muy constante salvo unos valores extremos al principio y al final, aquí con las descripciones anteriores tenemos que la desviación estándar es 0.4659 lo cual fue causada por esos valores extremos mencionados.\par
    \huge{f)}\normalsize \begin{Verbatim}[commandchars=\\\{\}]
{\color{incolor}In [{\color{incolor}47}]:} \PY{n}{plt}\PY{o}{.}\PY{n}{plot\PYZus{}date}\PY{p}{(}\PY{n}{x}\PY{o}{=}\PY{n}{dfsani}\PY{o}{.}\PY{n}{month}\PY{p}{,} \PY{n}{y}\PY{o}{=}\PY{n}{dfsani}\PY{o}{.}\PY{n}{Temp}\PY{p}{,} \PY{n}{fmt}\PY{o}{=}\PY{l+s+s2}{\PYZdq{}}\PY{l+s+s2}{b\PYZhy{}}\PY{l+s+s2}{\PYZdq{}}\PY{p}{)}
         \PY{n}{plt}\PY{o}{.}\PY{n}{title}\PY{p}{(}\PY{l+s+s2}{\PYZdq{}}\PY{l+s+s2}{Temperatura del agua}\PY{l+s+s2}{\PYZdq{}}\PY{p}{)}
         \PY{n}{plt}\PY{o}{.}\PY{n}{ylabel}\PY{p}{(}\PY{l+s+s2}{\PYZdq{}}\PY{l+s+s2}{°C}\PY{l+s+s2}{\PYZdq{}}\PY{p}{)}
         \PY{n}{plt}\PY{o}{.}\PY{n}{grid}\PY{p}{(}\PY{k+kc}{True}\PY{p}{)}
         \PY{n}{plt}\PY{o}{.}\PY{n}{show}\PY{p}{(}\PY{p}{)}
\end{Verbatim}


    \begin{center}
    \adjustimage{max size={0.9\linewidth}{0.9\paperheight}}{output_15_0.png}
    \end{center}
    { \hspace*{\fill} \\}
    En en la gráfica final de este tipo que es de temperatura vemos que también hubo cierto valor constante con máximos y mínimos muy extremos los cuales desestabilizan los datos estadísticos, de aquí se sabe que la desviación estándar es 0.5482, la cual es muy grande.\par
    \item Por ultimo producimos gráficas superpuestas con doble eje vertical (izquierda, derecha): g) Nivel de mar y Salinidad; h) Nivel de mar y Temperatura.\par
    \huge{g)}\normalsize\begin{Verbatim}[commandchars=\\\{\}]
{\color{incolor}In [{\color{incolor}50}]:} \PY{k+kn}{from} \PY{n+nn}{pylab} \PY{k}{import} \PY{n}{figure}\PY{p}{,} \PY{n}{show}\PY{p}{,} \PY{n}{legend}\PY{p}{,} \PY{n}{ylabel}
         \PY{n}{fig1} \PY{o}{=} \PY{n}{figure}\PY{p}{(}\PY{p}{)}
         
         \PY{n}{ax1} \PY{o}{=} \PY{n}{fig1}\PY{o}{.}\PY{n}{add\PYZus{}subplot}\PY{p}{(}\PY{l+m+mi}{111}\PY{p}{)}
         \PY{n}{line1} \PY{o}{=} \PY{n}{ax1}\PY{o}{.}\PY{n}{plot}\PY{p}{(}\PY{n}{dfsani}\PY{p}{[}\PY{l+s+s1}{\PYZsq{}}\PY{l+s+s1}{Sal}\PY{l+s+s1}{\PYZsq{}}\PY{p}{]}\PY{p}{,} \PY{l+s+s1}{\PYZsq{}}\PY{l+s+s1}{b\PYZhy{}}\PY{l+s+s1}{\PYZsq{}}\PY{p}{)}
         \PY{n}{ylabel}\PY{p}{(}\PY{l+s+s2}{\PYZdq{}}\PY{l+s+s2}{Salinidad}\PY{l+s+s2}{\PYZdq{}}\PY{p}{)}
         
         \PY{n}{ax2} \PY{o}{=} \PY{n}{fig1}\PY{o}{.}\PY{n}{add\PYZus{}subplot}\PY{p}{(}\PY{l+m+mi}{111}\PY{p}{,} \PY{n}{sharex}\PY{o}{=}\PY{n}{ax1}\PY{p}{,} \PY{n}{frameon}\PY{o}{=}\PY{k+kc}{False}\PY{p}{)}
         \PY{n}{line2} \PY{o}{=} \PY{n}{ax2}\PY{o}{.}\PY{n}{plot}\PY{p}{(}\PY{n}{dfsarg}\PY{p}{[}\PY{l+s+s1}{\PYZsq{}}\PY{l+s+s1}{Waterlevel}\PY{l+s+s1}{\PYZsq{}}\PY{p}{]}\PY{p}{,} \PY{l+s+s1}{\PYZsq{}}\PY{l+s+s1}{xr\PYZhy{}}\PY{l+s+s1}{\PYZsq{}}\PY{p}{)}
         \PY{n}{ax2}\PY{o}{.}\PY{n}{yaxis}\PY{o}{.}\PY{n}{tick\PYZus{}right}\PY{p}{(}\PY{p}{)}
         \PY{n}{ax2}\PY{o}{.}\PY{n}{yaxis}\PY{o}{.}\PY{n}{set\PYZus{}label\PYZus{}position}\PY{p}{(}\PY{l+s+s2}{\PYZdq{}}\PY{l+s+s2}{right}\PY{l+s+s2}{\PYZdq{}}\PY{p}{)}
         \PY{n}{ylabel}\PY{p}{(}\PY{l+s+s2}{\PYZdq{}}\PY{l+s+s2}{Nivel del agua}\PY{l+s+s2}{\PYZdq{}}\PY{p}{)}
          
         \PY{n}{show}\PY{p}{(}\PY{p}{)}
\end{Verbatim}


    \begin{center}
    \adjustimage{max size={0.9\linewidth}{0.9\paperheight}}{output_16_0.png}
    \end{center}
    { \hspace*{\fill} \\}
    \par
    \huge{h)}\normalsize\begin{Verbatim}[commandchars=\\\{\}]
{\color{incolor}In [{\color{incolor}51}]:} \PY{k+kn}{from} \PY{n+nn}{pylab} \PY{k}{import} \PY{n}{figure}\PY{p}{,} \PY{n}{show}\PY{p}{,} \PY{n}{legend}\PY{p}{,} \PY{n}{ylabel}
         \PY{n}{fig1} \PY{o}{=} \PY{n}{figure}\PY{p}{(}\PY{p}{)}
         
         \PY{n}{ax1} \PY{o}{=} \PY{n}{fig1}\PY{o}{.}\PY{n}{add\PYZus{}subplot}\PY{p}{(}\PY{l+m+mi}{111}\PY{p}{)}
         \PY{n}{line1} \PY{o}{=} \PY{n}{ax1}\PY{o}{.}\PY{n}{plot}\PY{p}{(}\PY{n}{dfsani}\PY{p}{[}\PY{l+s+s1}{\PYZsq{}}\PY{l+s+s1}{Temp}\PY{l+s+s1}{\PYZsq{}}\PY{p}{]}\PY{p}{,} \PY{l+s+s1}{\PYZsq{}}\PY{l+s+s1}{b\PYZhy{}}\PY{l+s+s1}{\PYZsq{}}\PY{p}{)}
         \PY{n}{ylabel}\PY{p}{(}\PY{l+s+s2}{\PYZdq{}}\PY{l+s+s2}{Temperatura}\PY{l+s+s2}{\PYZdq{}}\PY{p}{)}
         \PY{n}{ax2} \PY{o}{=} \PY{n}{fig1}\PY{o}{.}\PY{n}{add\PYZus{}subplot}\PY{p}{(}\PY{l+m+mi}{111}\PY{p}{,} \PY{n}{sharex}\PY{o}{=}\PY{n}{ax1}\PY{p}{,} \PY{n}{frameon}\PY{o}{=}\PY{k+kc}{False}\PY{p}{)}
         \PY{n}{line2} \PY{o}{=} \PY{n}{ax2}\PY{o}{.}\PY{n}{plot}\PY{p}{(}\PY{n}{dfsarg}\PY{p}{[}\PY{l+s+s1}{\PYZsq{}}\PY{l+s+s1}{Waterlevel}\PY{l+s+s1}{\PYZsq{}}\PY{p}{]}\PY{p}{,} \PY{l+s+s1}{\PYZsq{}}\PY{l+s+s1}{xr\PYZhy{}}\PY{l+s+s1}{\PYZsq{}}\PY{p}{)}
         \PY{n}{ax2}\PY{o}{.}\PY{n}{yaxis}\PY{o}{.}\PY{n}{tick\PYZus{}right}\PY{p}{(}\PY{p}{)}
         \PY{n}{ax2}\PY{o}{.}\PY{n}{yaxis}\PY{o}{.}\PY{n}{set\PYZus{}label\PYZus{}position}\PY{p}{(}\PY{l+s+s2}{\PYZdq{}}\PY{l+s+s2}{right}\PY{l+s+s2}{\PYZdq{}}\PY{p}{)}
         \PY{n}{ylabel}\PY{p}{(}\PY{l+s+s2}{\PYZdq{}}\PY{l+s+s2}{Nivel del agua}\PY{l+s+s2}{\PYZdq{}}\PY{p}{)}
         \PY{n}{xlim}\PY{o}{=} 
         \PY{n}{show}\PY{p}{(}\PY{p}{)}
\end{Verbatim}


    \begin{center}
    \adjustimage{max size={0.9\linewidth}{0.9\paperheight}}{output_17_0.png}
    \end{center}
    { \hspace*{\fill} \\}
    
    \begin{Verbatim}[commandchars=\\\{\}]
{\color{incolor}In [{\color{incolor}77}]:} \PY{n}{fig}\PY{p}{,} \PY{n}{ax1} \PY{o}{=} \PY{n}{plt}\PY{o}{.}\PY{n}{subplots}\PY{p}{(}\PY{p}{)}
         \PY{n}{date}\PY{o}{=}\PY{n}{dfsani}\PY{p}{[}\PY{l+s+s1}{\PYZsq{}}\PY{l+s+s1}{Ndate}\PY{l+s+s1}{\PYZsq{}}\PY{p}{]}
         \PY{n}{temp}\PY{o}{=}\PY{n}{dfsani}\PY{o}{.}\PY{n}{Temp}
         \PY{n}{WL}\PY{o}{=}\PY{n}{dfsarg}\PY{o}{.}\PY{n}{Waterlevel}
         \PY{n}{ax1}\PY{o}{.}\PY{n}{plot}\PY{p}{(}\PY{n}{date}\PY{p}{,}\PY{n}{temp}\PY{p}{,}\PY{l+s+s1}{\PYZsq{}}\PY{l+s+s1}{y\PYZhy{}}\PY{l+s+s1}{\PYZsq{}}\PY{p}{,} \PY{n}{label}\PY{o}{=}\PY{l+s+s1}{\PYZsq{}}\PY{l+s+s1}{Salinidad}\PY{l+s+s1}{\PYZsq{}}\PY{p}{)}\PY{p}{;} \PY{n}{plt}\PY{o}{.}\PY{n}{legend}\PY{p}{(}\PY{n}{loc}\PY{o}{=}\PY{l+s+s1}{\PYZsq{}}\PY{l+s+s1}{upper left}\PY{l+s+s1}{\PYZsq{}}\PY{p}{)}
         \PY{n}{ax1}\PY{o}{.}\PY{n}{set\PYZus{}xlabel}\PY{p}{(}\PY{l+s+s1}{\PYZsq{}}\PY{l+s+s1}{Tiempo}\PY{l+s+s1}{\PYZsq{}}\PY{p}{)}
         \PY{n}{ax1}\PY{o}{.}\PY{n}{set\PYZus{}ylabel}\PY{p}{(}\PY{l+s+s1}{\PYZsq{}}\PY{l+s+s1}{Salinidad(PPT)}\PY{l+s+s1}{\PYZsq{}}\PY{p}{)}
         \PY{n}{ax2} \PY{o}{=} \PY{n}{ax1}\PY{o}{.}\PY{n}{twinx}\PY{p}{(}\PY{p}{)}
         \PY{n}{ax2}\PY{o}{.}\PY{n}{plot}\PY{p}{(}\PY{n}{date}\PY{p}{,} \PY{n}{WL} \PY{p}{,} \PY{l+s+s1}{\PYZsq{}}\PY{l+s+s1}{c\PYZhy{}}\PY{l+s+s1}{\PYZsq{}}\PY{p}{,} \PY{n}{label}\PY{o}{=}\PY{l+s+s1}{\PYZsq{}}\PY{l+s+s1}{Nivel del mar}\PY{l+s+s1}{\PYZsq{}}\PY{p}{)}\PY{p}{;} \PY{n}{plt}\PY{o}{.}\PY{n}{legend}\PY{p}{(}\PY{n}{loc}\PY{o}{=}\PY{l+s+s1}{\PYZsq{}}\PY{l+s+s1}{upper right}\PY{l+s+s1}{\PYZsq{}}\PY{p}{)}
         \PY{n}{ax2}\PY{o}{.}\PY{n}{set\PYZus{}ylabel}\PY{p}{(}\PY{l+s+s1}{\PYZsq{}}\PY{l+s+s1}{Nivel del mar (m)}\PY{l+s+s1}{\PYZsq{}}\PY{p}{)}
         \PY{n}{fig}\PY{o}{.}\PY{n}{tight\PYZus{}layout}\PY{p}{(}\PY{p}{)}
         \PY{n}{plt}\PY{o}{.}\PY{n}{xlim}\PY{p}{(}\PY{p}{(}\PY{l+s+s2}{\PYZdq{}}\PY{l+s+s2}{10/26/2017 13:00:00}\PY{l+s+s2}{\PYZdq{}}\PY{p}{,}\PY{l+s+s2}{\PYZdq{}}\PY{l+s+s2}{ 10/31/2017 13:00:00}\PY{l+s+s2}{\PYZdq{}}\PY{p}{)}\PY{p}{)}
         \PY{n}{plt}\PY{o}{.}\PY{n}{show}\PY{p}{(}\PY{p}{)}
\end{Verbatim}
    \begin{center}
    \adjustimage{max size={0.9\linewidth}{0.9\paperheight}}{output_18_0.png}
    \end{center}
    { \hspace*{\fill} \\}
    Donde estas gráficas enfocándonos bien en un lapso de 5 días del 27 al 31 de octubre se ve claro el comportamiento de estos y como su relación en los dos casos si es notoria, en primera, la salinidad del agua bajara si el nivel del agua aumenta, esto por la concentración de la sal, ya que la sal se distribuirá en un mayor volumen de agua.\par
    También podemos ver que la temperatura de igual manera descenderá si el volumen aumenta ya que mientras mas volumen de agua haya, mas energía se necesitara para calentarla, entonces como los cambios de temperatura no son lineales se da lugar a este fenómeno en la temperatura.
    
    \end{enumerate}
    \bibliographystyle{plain}
    \bibliography{biblio.bib}
    
    \end{document}
